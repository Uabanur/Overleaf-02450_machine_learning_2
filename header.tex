\usepackage[utf8]{inputenc}

\usepackage[english]{babel}
\usepackage{parskip}
\usepackage{tablefootnote}
\usepackage[nottoc]{tocbibind}
\usepackage{amssymb}
\usepackage{mathtools} 
\usepackage{geometry}
 \geometry{
 a4paper,
 total={140mm,227mm},
 left=32mm,
 top=35mm,
 }

\usepackage{float}
\usepackage{graphicx}
\usepackage{epstopdf}
\usepackage{amsmath}
\usepackage{bm}

%dank
\usepackage{marvosym}

\setlength{\parindent}{0pt}
\numberwithin{figure}{section}
% equation number format: in section 1, (1.1), (1.2) ..
\usepackage{amsmath}
\numberwithin{equation}{section}

\makeatletter
% we use \prefix@<level> only if it is defined
\renewcommand{\@seccntformat}[1]{%
  \ifcsname prefix@#1\endcsname
    \csname prefix@#1\endcsname
  \else
    \csname the#1\endcsname\quad
  \fi}
% define \prefix@section
% \newcommand\prefix@section{}
\makeatother

%% multi col
\usepackage{multicol}

\usepackage{siunitx}

\usepackage[colorlinks]{hyperref}

% hyperlinks
\usepackage{hyperref}
\hypersetup{
    colorlinks=true,
    linkcolor=black,
    citecolor=red,
    urlcolor=blue,
    filecolor=red,
    bookmarksopen=true
} 
\urlstyle{same}


% colored frames (boxes)
\usepackage{xcolor}
\usepackage{mdframed}

% vector arrow with \vv{ }
\usepackage{esvect}

%\hyphenpenalty 10000
%\exhyphenpenalty 10000
%\usepackage[none]{hyphenat}
%\usepackage[document]{ragged2e}

% strike through text
\usepackage[normalem]{ulem}
% \sout{Hello World}

\usepackage{listings}
\lstset{language=Java,
basicstyle=\ttfamily,
keywordstyle=\color{blue}\ttfamily,
stringstyle=\color{red}\ttfamily,
showstringspaces=false,
commentstyle=\color{purple}\ttfamily,
morecomment=[l][\color{magenta}]{\#}, 
%numbers=left,
numbersep=5pt, 
backgroundcolor=\color{white}, 
breaklines=true
}

\usepackage{siunitx}
%\usepackage{unixode}


% spacing in enum and itemize
\usepackage{enumitem}

% \usepackage{slantsc}
\usepackage{bold-extra}
% New commands 



\newcommand{\C}{\textcircled{}}
\newcommand{\CT}{\textcircled{$\checkmark$}}

\usepackage{caption}
\usepackage{subcaption}

% Til linjedeling i en tabelcelle
\usepackage{makecell}

% Vektor
\newcommand{\vek}[1]{\bm{#1}}

% Norm (|| ||)
\newcommand{\norm}[1]{\left\lVert#1\right\rVert}


%% COLORED MATRIX
%\usepackage{tikz}
\usepackage{collcell}

 %The min, mid and max values
\newcommand*{\MinNumber}{0.0}%
\newcommand*{\MidNumber}{0.5} %
\newcommand*{\MaxNumber}{1.0}%

%Apply the gradient macro
\newcommand{\ApplyGradient}[1]{%
        \ifdim #1 pt > \MidNumber pt
            \pgfmathsetmacro{\PercentColor}{max(min(100.0*(#1 - \MidNumber)/(\MaxNumber-\MidNumber),100.0),0.00)} %
            \hspace{-0.33em}\colorbox{green!\PercentColor!yellow}{#1}
        \else
            \pgfmathsetmacro{\PercentColor}{max(min(100.0*(\MidNumber - #1)/(\MidNumber-\MinNumber),100.0),0.00)} %
            \hspace{-0.33em}\colorbox{red!\PercentColor!yellow}{#1}
        \fi
}

\newcolumntype{R}{>{\collectcell\ApplyGradient}c<{\endcollectcell}}
\renewcommand{\arraystretch}{0}
\setlength{\fboxsep}{3mm} % box size
\setlength{\tabcolsep}{0pt}



%%%%%%%%%%%%%%%%%%
%% HEADER OVER %%%
%%%%%%%%%%%%%%%%%%