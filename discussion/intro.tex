As mentioned in report 1, the original problem the data set was intended to solve was identification of glass fragments at crime scenes. The original data analysis by Vina Spiehler\footnote{\url{ https://archive.ics.uci.edu/ml/datasets/glass+identification}} solved a simplified version of the problem: Spiehler filtered the data set of 214 observations to only contain type 1 (\texttt{building windows glass, float processed}), type 2 (\texttt{building windows, non float}) and type 3 (\texttt{vehicle windows, float}). Spiehler then grouped type 1 and type 2 into a class \texttt{float processed}, and renamed type 3 to \texttt{non-float processed}. Spiehler's task was then a \textbf{binary classification problem} with 163 observations. Classifying the data set using BEAGLE yielded an error rate of $\approx $ 18 \%. Considering that the classification problem solved in this report has 7 classes\footnote{or 6 classes, neglecting type 4 with no observations}, and the generalization errors were at best 19 \% (KNN) and at worst 28.6 \% (ANN with no regularization), the machine learning methods acquired in the second section of the course  successfully solved relevant classification and regression problems for the data set.